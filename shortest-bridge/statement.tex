\documentclass{article}

\usepackage{geometry}
\usepackage{amsmath}
\usepackage{graphicx, eso-pic}
\usepackage{listings}
\usepackage{hyperref}
\usepackage{multicol}
\usepackage{fancyhdr}
\pagestyle{fancy}
\fancyhf{}
\hypersetup{ colorlinks=true, linkcolor=black, filecolor=magenta, urlcolor=cyan}
\geometry{ a4paper, total={170mm,257mm}, top=40mm, right=20mm, bottom=20mm, left=20mm}
\setlength{\parindent}{0pt}
\setlength{\parskip}{0.3em}
\renewcommand{\headrulewidth}{0pt}
\rfoot{\thepage}
\lfoot{Seleksi Asisten Laboratorium Pemrograman STEI ITB 2022/2023}
\lstset{
    basicstyle=\ttfamily\small,
    columns=fixed,
    extendedchars=true,
    breaklines=true,
    tabsize=2,
    prebreak=\raisebox{0ex}[0ex][0ex]{\ensuremath{\hookleftarrow}},
    frame=none,
    showtabs=false,
    showspaces=false,
    showstringspaces=false,
    prebreak={},
    keywordstyle=\color[rgb]{0.627,0.126,0.941},
    commentstyle=\color[rgb]{0.133,0.545,0.133},
    stringstyle=\color[rgb]{01,0,0},
    captionpos=t,
    escapeinside={(\%}{\%)}
}

\begin{document}

\begin{center}
    \section*{Menyatukan Dua Pulau} % ganti judul soal

    \begin{tabular}{ | c c | }
        \hline
        Batas Waktu  & 2s \\    % jangan lupa ganti time limit
        Batas Memori & 128MB \\  % jangan lupa ganti memory limit
        \hline
    \end{tabular}
\end{center}

\subsection*{Deskripsi}
Diberikan sebuah matrix grid berukuran n x n dengan tiap elemen mungkin bernilai 0 (air) atau 1 (daratan). Dalam sebuah grid terdapat tepat 2 pulau, yang terdiri dari kumpulan daratan yang bertetanggaan (atas, bawah, kiri, kanan). 

Tentukanlah jumlah terkecil elemen air (0) yang harus diisi dengan daratan (1) supaya kedua pulau tersambung menjadi satu. 



\subsection*{Format Masukan}
Baris Pertama, berisi bilangan N, yang menyatakan baris dan kolom dari matrix.
N baris berikutnya berisi N buah bilangan bulat yang menyatakan tiap elemen dalam matrix.


\subsection*{Format Keluaran}
Sebuah bilangan bulat yang menyatakan jumlah terkecil elemen air (0) yang harus diisi dengan daratan (1) untuk menyatukan kedua pulau.

\subsection*{Batasan Masukan}
\begin{itemize}
 \item $2 \leq  N  \leq 100$
\end{itemize}

\linebreak
\begin{multicols}{2}
\subsection*{Contoh Masukan 1}
\begin{lstlisting}
2
1 0
0 1
\end{lstlisting}
\columnbreak
\subsection*{Contoh Keluaran 1}
\begin{lstlisting}
1
\end{lstlisting}
\vfill
\end{multicols}

\subsection*{Penjelasan 1}
Dengan mengubah salah satu elemen yang bernilai 0 menjadi 1, kedua pulau tersambung.

\linebreak
\begin{multicols}{2}
\subsection*{Contoh Masukan 2}
\begin{lstlisting}
3
0 1 0
0 0 0
0 0 1
\end{lstlisting}
\columnbreak
\subsection*{Contoh Keluaran 2}
\begin{lstlisting}
2
\end{lstlisting}
\vfill
\end{multicols}

\subsection*{Penjelasan}
Kondisi akhir:
\begin{lstlisting}
0 1 0
0 1 0
0 1 1
\end{lstlisting}

\end{document}