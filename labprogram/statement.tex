\documentclass{article}

\usepackage{geometry}
\usepackage{amsmath}
\usepackage{graphicx, eso-pic}
\usepackage{listings}
\usepackage{hyperref}
\usepackage{multicol}
\usepackage{fancyhdr}
\pagestyle{fancy}
\fancyhf{}
\hypersetup{ colorlinks=true, linkcolor=black, filecolor=magenta, urlcolor=cyan}
\geometry{ a4paper, total={170mm,257mm}, top=40mm, right=20mm, bottom=20mm, left=20mm}
\setlength{\parindent}{0pt}
\setlength{\parskip}{0.3em}
\renewcommand{\headrulewidth}{0pt}
\rfoot{\thepage}
\lfoot{Seleksi Asisten Laboratorium Pemrograman STEI ITB 2022/2023}
\lstset{
    basicstyle=\ttfamily\small,
    columns=fixed,
    extendedchars=true,
    breaklines=true,
    tabsize=2,
    prebreak=\raisebox{0ex}[0ex][0ex]{\ensuremath{\hookleftarrow}},
    frame=none,
    showtabs=false,
    showspaces=false,
    showstringspaces=false,
    prebreak={},
    keywordstyle=\color[rgb]{0.627,0.126,0.941},
    commentstyle=\color[rgb]{0.133,0.545,0.133},
    stringstyle=\color[rgb]{01,0,0},
    captionpos=t,
    escapeinside={(\%}{\%)}
}

\begin{document}

\begin{center}
    \section*{LABPROGRAM} % ganti judul soal

    \begin{tabular}{ | c c | }
        \hline
        Batas Waktu  & 1s \\    % jangan lupa ganti time limit
        Batas Memori & 128MB \\  % jangan lupa ganti memory limit
        \hline
    \end{tabular}
\end{center}

\subsection*{Deskripsi}
Diberikan satu buah string acak S sebagai masukan. Dari huruf-huruf yang ada pada string tersebut mungkin dapat dibentuk kata `LABPROGRAM`. Hitunglah berapa kata `LABPROGRAM` yang dapat dibentuk!

\subsection*{Format Masukan}
Masukan hanya terdiri dari satu baris. Baris pertama adalah sebuah string acak S yang hanya terdiri dari huruf besar alfabet.

\subsection*{Format Keluaran}
Keluaran terdiri dari satu baris yang merupakan sebuah integer yang merepresentasikan jumlah kata `LABPROGRAM` yang dapat dibentuk.

\subsection*{Batasan Masukan}
\begin{itemize}
 \item $1 \leq |S|  \leq 10^7$
\end{itemize}

\begin{multicols}{2}
\subsection*{Contoh Masukan}
\begin{lstlisting}
AXRHLDAYJRMLITLDAEPWBRBUXYPEZOGRZGOZRAM
\end{lstlisting}
\null
\columnbreak
\subsection*{Contoh Keluaran}
\begin{lstlisting}
2
\end{lstlisting}
\vfill
\null
\end{multicols}

\subsection*{Penjelasan}
Terdapat dua kata `LABPROGRAM` yang dapat dibentuk dari string tersebut. Huruf-hurufnya ditandai dengan cetak \textbf{tebal}.
\begin{enumerate}
    \item \textbf{A}X\textbf{R}H\textbf{L}D\textbf{A}YJR\textbf{M}LITLDAE\textbf{P}W\textbf{BR}BUXYPEZ\textbf{OG}RZGOZRAM
    \item AXRHLDAYJRM\textbf{L}ITLD\textbf{A}EPWBR\textbf{B}UXY\textbf{P}EZOG\textbf{R}Z\textbf{GO}Z\textbf{RAM}
\end{enumerate}
\end{document}