\documentclass{article}

\usepackage{geometry}
\usepackage{amsmath}
\usepackage{graphicx, eso-pic}
\usepackage{listings}
\usepackage{hyperref}
\usepackage{multicol}
\usepackage{fancyhdr}
\pagestyle{fancy}
\fancyhf{}
\hypersetup{ colorlinks=true, linkcolor=black, filecolor=magenta, urlcolor=cyan}
\geometry{ a4paper, total={170mm,257mm}, top=40mm, right=20mm, bottom=20mm, left=20mm}
\setlength{\parindent}{0pt}
\setlength{\parskip}{0.3em}
\renewcommand{\headrulewidth}{0pt}
\rfoot{\thepage}
\lfoot{Seleksi Asisten Laboratorium Pemrograman STEI ITB 2022/2023}
\lstset{
    basicstyle=\ttfamily\small,
    columns=fixed,
    extendedchars=true,
    breaklines=true,
    tabsize=2,
    prebreak=\raisebox{0ex}[0ex][0ex]{\ensuremath{\hookleftarrow}},
    frame=none,
    showtabs=false,
    showspaces=false,
    showstringspaces=false,
    prebreak={},
    keywordstyle=\color[rgb]{0.627,0.126,0.941},
    commentstyle=\color[rgb]{0.133,0.545,0.133},
    stringstyle=\color[rgb]{01,0,0},
    captionpos=t,
    escapeinside={(\%}{\%)}
}

\begin{document}

\begin{center}
    \section*{Mengubah Rentang Dinamis} % ganti judul soal

    \begin{tabular}{ | c c | }
        \hline
        Batas Waktu  & 1s \\    % jangan lupa ganti time limit
        Batas Memori & 256MB \\  % jangan lupa ganti memory limit
        \hline
    \end{tabular}
\end{center}

\subsection*{Deskripsi}
Diberikan array bilangan asli dengan ukuran N, diberikan juga Q perintah yang harus dilakukan. Ada 2 tipe perintah:
\begin{enumerate}
    \item Tipe 1, diberikan bilangan K, keluarkan nilai array pada posisi ke-K!
    \item Tipe 2, diberikan tiga bilangan A, B dan X, Untuk setiap nilai array dalam rentang (inklusif) [A, B] tambahkan dengan nilai P(X), dimana P(X) menyatakan bilangan prima ke-X!
\end{enumerate}

\subsection*{Format Masukan}
Baris Pertama, berisi bilangan N dan Q, yang masing-masing menyatakan banyaknya array dan perintah
Baris kedua, berisi N bilangan asli $AR_1, …, AR_N$, menyatakan nilai-nilai dari array N bilangan tersebut.
Q baris selanjutnya berisi dua tipe perintah:
\begin{itemize}
    \item 1 K, untuk tipe pertama
    \item 2 A B X, untuk tipe kedua
\end{itemize}


\subsection*{Format Keluaran}
Keluaran berupa sebuah bilangan untuk setiap keluaran perintah tipe pertama.

\subsection*{Batasan Masukan}
\begin{itemize}
 \item $1 \leq  N, Q, K  \leq 2 \cdot 10^5$
 \item $1 \leq  X, {AR_i}  \leq 10^5$
 \item $1 \leq  A \leq B  \leq N$
 \item $1 \leq  K \leq N$
\end{itemize}

\linebreak
\begin{multicols}{2}
\subsection*{Contoh Masukan 1}
\begin{lstlisting}
8 3
3 2 4 5 1 1 5 3
1 4 
2 2 5 1
1 3


\end{lstlisting}
\null
\columnbreak
\subsection*{Contoh Keluaran 1}
\begin{lstlisting}
5
6
\end{lstlisting}
\vfill
\null
\end{multicols}

\subsection*{Penjelasan}
Terdapat 3 perintah:
\begin{itemize}
    \item 1 4, karena tipe perintah pertama, dikeluarkan nilai array pada posisi 4 yaitu 5
    \item 2 2 5 1, karena tipe perintah kedua, tambahkan nilai P(1) yaitu 2, dari posisi array 2 sampai 5, sehingga isi array sekarang: [3, 4, 6, 7, 3, 3, 5, 3]
    \item 1 3, karena tipe perintah pertama, dikeluarkan nilai array pada posisi 3 yaitu 6
\end{itemize}

\end{document}