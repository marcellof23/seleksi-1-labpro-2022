\documentclass{article}

\usepackage{geometry}
\usepackage{amsmath}
\usepackage{graphicx, eso-pic}
\usepackage{listings}
\usepackage{hyperref}
\usepackage{multicol}
\usepackage{fancyhdr}
\pagestyle{fancy}
\fancyhf{}
\hypersetup{ colorlinks=true, linkcolor=black, filecolor=magenta, urlcolor=cyan}
\geometry{ a4paper, total={170mm,257mm}, top=40mm, right=20mm, bottom=20mm, left=20mm}
\setlength{\parindent}{0pt}
\setlength{\parskip}{0.3em}
\renewcommand{\headrulewidth}{0pt}
\rfoot{\thepage}
\lfoot{Seleksi Asisten Laboratorium Pemrograman STEI ITB 2022/2023}
\lstset{
    basicstyle=\ttfamily\small,
    columns=fixed,
    extendedchars=true,
    breaklines=true,
    tabsize=2,
    prebreak=\raisebox{0ex}[0ex][0ex]{\ensuremath{\hookleftarrow}},
    frame=none,
    showtabs=false,
    showspaces=false,
    showstringspaces=false,
    prebreak={},
    keywordstyle=\color[rgb]{0.627,0.126,0.941},
    commentstyle=\color[rgb]{0.133,0.545,0.133},
    stringstyle=\color[rgb]{01,0,0},
    captionpos=t,
    escapeinside={(\%}{\%)}
}

\begin{document}

\begin{center}
    \section*{Menghitung XOR} % ganti judul soal

    \begin{tabular}{ | c c | }
        \hline
        Batas Waktu  & 1s \\    % jangan lupa ganti time limit
        Batas Memori & 256MB \\  % jangan lupa ganti memory limit
        \hline
    \end{tabular}
\end{center}

\subsection*{Deskripsi}
Diberikan Q buah perintah, setiap perintah Anda diberikan dua buah bilangan L dan R yang menyatakan rentang bilangan asli L sampai R (inklusif). Untuk setiap perintah, hitunglah nilai hasil \href{https://en.wikipedia.org/wiki/Exclusive_or}{XOR} dari L sampai R!

\subsection*{Format Masukan}
Baris pertama, berisi bilangan Q yang menyatakan banyaknya perintah. Q baris selanjutnya, berisi dua buah bilangan L dan R yang menyatakan rentang XOR.

\subsection*{Format Keluaran}
Keluaran berupa Q buah baris yang masing-masing berisi satu bilangan yaitu hasil XOR dari L sampai R.

\subsection*{Batasan Masukan}
\begin{itemize}
 \item $1 \leq Q  \leq 2 \cdot 10^5$
 \item $1 \leq L  \leq R  \leq 10^{18}$
\end{itemize}

\linebreak
\begin{multicols}{2}
\subsection*{Contoh Masukan 1}
\begin{lstlisting}
2
1 2
7 10

\end{lstlisting}
\null
\columnbreak
\subsection*{Contoh Keluaran 1}
\begin{lstlisting}
3
12

\end{lstlisting}
\vfill
\null
\end{multicols}

\subsection*{Penjelasan}
1 XOR 2 = 01 XOR 10 = 11 (basis 2) = 3 (basis 10) \\
7 XOR 8 XOR 9 XOR 10 = 0111 XOR 1000 XOR 1001 XOR 1010 = 1100 (basis 2) = 12 (basis 10) \\

\end{document}