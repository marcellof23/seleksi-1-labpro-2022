\documentclass{article}

\usepackage{geometry}
\usepackage{amsmath}
\usepackage{graphicx, eso-pic}
\usepackage{listings}
\usepackage{hyperref}
\usepackage{multicol}
\usepackage{fancyhdr}
\pagestyle{fancy}
\fancyhf{}
\hypersetup{ colorlinks=true, linkcolor=black, filecolor=magenta, urlcolor=cyan}
\geometry{ a4paper, total={170mm,257mm}, top=40mm, right=20mm, bottom=20mm, left=20mm}
\setlength{\parindent}{0pt}
\setlength{\parskip}{0.3em}
\renewcommand{\headrulewidth}{0pt}
\rfoot{\thepage}
\lfoot{Seleksi Asisten Laboratorium Pemrograman STEI ITB 2022/2023}
\lstset{
    basicstyle=\ttfamily\small,
    columns=fixed,
    extendedchars=true,
    breaklines=true,
    tabsize=2,
    prebreak=\raisebox{0ex}[0ex][0ex]{\ensuremath{\hookleftarrow}},
    frame=none,
    showtabs=false,
    showspaces=false,
    showstringspaces=false,
    prebreak={},
    keywordstyle=\color[rgb]{0.627,0.126,0.941},
    commentstyle=\color[rgb]{0.133,0.545,0.133},
    stringstyle=\color[rgb]{01,0,0},
    captionpos=t,
    escapeinside={(\%}{\%)}
}

\begin{document}

\begin{center}
    \section*{Kurung Seimbang} % ganti judul soal

    \begin{tabular}{ | c c | }
        \hline
        Batas Waktu  & 1s \\    % jangan lupa ganti time limit
        Batas Memori & 8MB \\  % jangan lupa ganti memory limit
        \hline
    \end{tabular}
\end{center}

\subsection*{Deskripsi}
Tanda kurung terdiri dari 3 jenis, yaitu pasangan “()”, “\{\}”, dan “[]”. Kita definisikan bahwa suatu string disebut sebagai kurung yang seimbang jika:
Kurung pembukanya berpasangan dengan kurung penutup.
Isi dari kedua kurung tersebut juga kurung yang seimbang.

Sebagai contoh “[()]” merupakan kurung yang seimbang. Sedangkan “\{[\}\}” bukan kurung yang seimbang.

DIberikan N buah string, print 1 jika string merupakan kurung yang seimbang, print 0 jika sebaliknya.


\subsection*{Format Masukan}
Baris pertama berisi satu bilangan bulat N, menyatakan banyaknya string
N baris selanjutnya berisi string yang akan diolah



\subsection*{Format Keluaran}
Keluarkan 1 atau 0 untuk tiap string
\subsection*{Batasan Masukan}
\begin{itemize}
 \item $1 \leq N  \leq 10^3$
 \item $0 \leq |s|  \leq 10^{3}$ dengan $|s|$ merupakan panjang dari string $s$
\end{itemize}

\linebreak
\begin{multicols}{2}
\subsection*{Contoh Masukan 1}
\begin{lstlisting}
2
{{}}
{[]}()


\end{lstlisting}
\null
\columnbreak
\subsection*{Contoh Keluaran 1}
\begin{lstlisting}
1
1


\end{lstlisting}
\vfill
\null
\end{multicols}

\subsection*{Penjelasan}

\end{document}
