\documentclass{article}

\usepackage{geometry}
\usepackage{amsmath}
\usepackage{graphicx, eso-pic}
\usepackage{listings}
\usepackage{hyperref}
\usepackage{multicol}
\usepackage{fancyhdr}
\usepackage{mathtools}

\DeclarePairedDelimiter\floor{\lfloor}{\rfloor}

\pagestyle{fancy}
\fancyhf{}
\hypersetup{ colorlinks=true, linkcolor=black, filecolor=magenta, urlcolor=cyan}
\geometry{ a4paper, total={170mm,257mm}, top=20mm, right=20mm, bottom=20mm, left=20mm}
\lhead{Seleksi Tahap 1 Laboratorium Pemrograman 2019}
\setlength{\parindent}{0pt}
\setlength{\parskip}{0.3em}
\renewcommand{\headrulewidth}{0pt}
\rfoot{\thepage}
\lfoot{Seleksi Tahap 1 Labpro 2019}
\lstset{
    basicstyle=\ttfamily\small,
    columns=fixed,
    extendedchars=true,
    breaklines=true,
    tabsize=2,
    prebreak=\raisebox{0ex}[0ex][0ex]{\ensuremath{\hookleftarrow}},
    frame=none,
    showtabs=false,
    showspaces=false,
    showstringspaces=false,
    prebreak={},
    keywordstyle=\color[rgb]{0.627,0.126,0.941},
    commentstyle=\color[rgb]{0.133,0.545,0.133},
    stringstyle=\color[rgb]{01,0,0},
    captionpos=t,
    escapeinside={(\%}{\%)}
}

\begin{document}

\begin{center}
    \section*{Banyak Anagram} % ganti judul soal

    \begin{tabular}{ | c c | }
        \hline
        Batas Waktu  & 2s \\    % jangan lupa ganti time limit
        Batas Memori & 256MB \\  % jangan lupa ganti memory limit
        \hline
    \end{tabular}
\end{center}

\subsection*{Deskripsi}

Diberikan $N$ kata, $S_1$, $S_2$, $…$, $S_N$, setiap kata ($S_i$) merupakan sebuah string yang terdiri dari huruf kecil alfabet. Tentukanlah berapa banyak \href{https://en.wikipedia.org/wiki/Anagram}{anagram} unik dari $N$ buah kata tersebut.

\subsection*{Format Masukan}
\begin{itemize}
\item Baris pertama, berisi satu bilangan bulat $N$, menyatakan banyaknya kata
\item $N$ baris berikutnya, berisi informasi $S_i$ yakni masukkan berupa string kata ke-$i$
\end{itemize}

\subsection*{Format Keluaran}
Keluarkan bilangan bulat berupa jawaban dari soal.

\subsection*{Batasan Input}

\begin{itemize}
    \item{$1 \leq N \leq 10^5$}
    \item{$1 \leq |S_i| \leq 20$}
\end{itemize}

\begin{multicols}{2}
\subsection*{Contoh Masukan}
\begin{lstlisting}
5
kasur
rusak
dean
edan
hasan

\end{lstlisting}
\columnbreak
\subsection*{Contoh Keluaran}
\begin{lstlisting}
3
\end{lstlisting}
\vfill
\null
\end{multicols}

\subsection*{Penjelasan}

Terdapat 3 anagram, $anagram_1 = \{kasur, rusak\}$, $anagram_2 = \{dean, edan\}$, $anagram_3 = \{hasan\}$.

\end{document}