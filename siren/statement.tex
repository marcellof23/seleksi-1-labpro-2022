\documentclass{article}

\usepackage{geometry}
\usepackage{amsmath}
\usepackage{graphicx, eso-pic}
\usepackage{listings}
\usepackage{hyperref}
\usepackage{multicol}
\usepackage{fancyhdr}
\pagestyle{fancy}
\fancyhf{}
\hypersetup{ colorlinks=true, linkcolor=black, filecolor=magenta, urlcolor=cyan}
\geometry{ a4paper, total={170mm,257mm}, top=40mm, right=20mm, bottom=20mm, left=20mm}
\setlength{\parindent}{0pt}
\setlength{\parskip}{0.3em}
\renewcommand{\headrulewidth}{0pt}
\rfoot{\thepage}
\lfoot{Seleksi Asisten Laboratorium Pemrograman STEI ITB 2022/2023}
\lstset{
    basicstyle=\ttfamily\small,
    columns=fixed,
    extendedchars=true,
    breaklines=true,
    tabsize=2,
    prebreak=\raisebox{0ex}[0ex][0ex]{\ensuremath{\hookleftarrow}},
    frame=none,
    showtabs=false,
    showspaces=false,
    showstringspaces=false,
    prebreak={},
    keywordstyle=\color[rgb]{0.627,0.126,0.941},
    commentstyle=\color[rgb]{0.133,0.545,0.133},
    stringstyle=\color[rgb]{01,0,0},
    captionpos=t,
    escapeinside={(\%}{\%)}
}

\begin{document}

\begin{center}
    \section*{Siren} % ganti judul soal

    \begin{tabular}{ | c c | }
        \hline
        Batas Waktu  & 1s \\    % jangan lupa ganti time limit
        Batas Memori & 64MB \\  % jangan lupa ganti memory limit
        \hline
    \end{tabular}
\end{center}

\subsection*{Deskripsi}
Seorang klien menerima sebuah {\it data stream} berupa bit acak pada setiap interval. Bit-bit tersebut kemudian diolah menjadi sebuah string. Klien tersebut ingin membuat kumpulan bit tersebut menjadi sebuah kumpulan bit yang saling berselang-seling. Klien tersebut ingin proses perubahannya terjadi dengan cepat dan efisien. Kalian yang berperan sebagai {\it engineer} akan memenuhi permintaan tersebut. Buatlah sebuah program untuk mengubah kumpulan bit acak tersebut menjadi bit yang saling berselang-seling dengan perubahan bit yang minimal, kemudian cetak berapa kali perubahan minimal tersebut ke layar!

\subsection*{Format Masukan}
Masukan hanya terdiri dari satu baris. Baris pertama adalah sebuah string S yang merepresentasikan sebuah {\it bit stream} acak.

\subsection*{Format Keluaran}
Keluaran terdiri dari satu baris yang merupakan sebuah integer yang merepresentasikan berapa kali bit diubah secara minimal agar menjadi selang-seling.

\subsection*{Batasan Masukan}
\begin{itemize}
 \item $0 \leq |S|  \leq 10^5$
\end{itemize}

\begin{multicols}{2}
\subsection*{Contoh Masukan 1}
\begin{lstlisting}
111000
\end{lstlisting}
\null
\columnbreak
\subsection*{Contoh Keluaran 1}
\begin{lstlisting}
2
\end{lstlisting}
\vfill
\null
\end{multicols}

\begin{multicols}{2}
\subsection*{Contoh Masukan 2}
\begin{lstlisting}
000111
\end{lstlisting}
\null
\columnbreak
\subsection*{Contoh Keluaran 2}
\begin{lstlisting}
2
\end{lstlisting}
\vfill
\null
\end{multicols}

\begin{multicols}{2}
\subsection*{Contoh Masukan 3}
\begin{lstlisting}
11010101101010111010101
\end{lstlisting}
\null
\columnbreak
\subsection*{Contoh Keluaran 3}
\begin{lstlisting}
8
\end{lstlisting}
\vfill
\null
\end{multicols}

\pagebreak

\subsection*{Penjelasan}
Bit yang diubah ditandai dengan cetak \textbf{tebal}.
\newline \newline
Contoh 1\newline
Input: 111000\newline
Proses:
\begin{enumerate}
 \item 1\textbf{1}1000
 \item 1010\textbf{0}0
 \item 101010
\end{enumerate}
Catatan: Hasil akhirnya bisa saja menjadi 010101, tetapi proses perubahannya lebih banyak daripada proses di atas.\newline

Contoh 2\newline
Input: 000111\newline
Proses:
\begin{enumerate}
 \item 0\textbf{0}0111
 \item 0101\textbf{1}1
 \item 010101
\end{enumerate}

Contoh 3\newline
Input: 11010101101010111010101\newline
Proses:
\begin{enumerate}
 \item 1\textbf{1}010101101010111010101
 \item 10\textbf{0}10101101010111010101
 \item 101\textbf{1}0101101010111010101 
 \item 1010\textbf{0}101101010111010101
 \item 10101\textbf{1}01101010111010101
 \item 101010\textbf{0}1101010111010101
 \item 1010101\textbf{1}101010111010101
 \item 101010101010101\textbf{1}1010101
 \item 10101010101010101010101
\end{enumerate}

\end{document}