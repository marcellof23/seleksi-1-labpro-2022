\documentclass{article}

\usepackage{geometry}
\usepackage{amsmath}
\usepackage{graphicx, eso-pic}
\usepackage{listings}
\usepackage{hyperref}
\usepackage{multicol}
\usepackage{fancyhdr}
\pagestyle{fancy}
\fancyhf{}
\hypersetup{ colorlinks=true, linkcolor=black, filecolor=magenta, urlcolor=cyan}
\geometry{ a4paper, total={170mm,257mm}, top=40mm, right=20mm, bottom=20mm, left=20mm}
\setlength{\parindent}{0pt}
\setlength{\parskip}{0.3em}
\renewcommand{\headrulewidth}{0pt}
\rfoot{\thepage}
\lfoot{Seleksi Asisten Laboratorium Pemrograman STEI ITB 2022/2023}
\lstset{
    basicstyle=\ttfamily\small,
    columns=fixed,
    extendedchars=true,
    breaklines=true,
    tabsize=2,
    prebreak=\raisebox{0ex}[0ex][0ex]{\ensuremath{\hookleftarrow}},
    frame=none,
    showtabs=false,
    showspaces=false,
    showstringspaces=false,
    prebreak={},
    keywordstyle=\color[rgb]{0.627,0.126,0.941},
    commentstyle=\color[rgb]{0.133,0.545,0.133},
    stringstyle=\color[rgb]{01,0,0},
    captionpos=t,
    escapeinside={(\%}{\%)}
}

\begin{document}

\begin{center}
    \section*{Subarray yang Setara} % ganti judul soal

    \begin{tabular}{ | c c | }
        \hline
        Batas Waktu  & 2s \\    % jangan lupa ganti time limit
        Batas Memori & 128MB \\  % jangan lupa ganti memory limit
        \hline
    \end{tabular}
\end{center}

\subsection*{Deskripsi}
Pada suatu array bilangan bulat positif dengan panjang n, derajatnya adalah frekuensi terbesar dari elemen manapun dalam array tersebut.

Tentukanlah panjang dari subarray (kontigu) terkecil yang memiliki derajat yang sama dengan array masukan.


\subsection*{Format Masukan}
Baris Pertama, berisi bilangan N, yang menyatakan jumlah elemen dalam array.
Baris kedua, berisi N bilangan bulat $E_1, …, E_N$, menyatakan  elemen-elemen dalam array.


\subsection*{Format Keluaran}
Keluaran berupa sebuah bilangan yang menyatakan panjang dari subarray terkecil yang memiliki derajat yang sama dengan array masukan.

\subsection*{Batasan Masukan}
\begin{itemize}
 \item $1 \leq  N  \leq 1000$
 \item $1 \leq  {E_i}  \leq 50$
\end{itemize}

\linebreak
\begin{multicols}{2}
\subsection*{Contoh Masukan 1}
\begin{lstlisting}
7
1 2 2 3 1 4 2


\end{lstlisting}
\null
\columnbreak
\subsection*{Contoh Keluaran 1}
\begin{lstlisting}
6
\end{lstlisting}
\vfill
\null
\end{multicols}

\subsection*{Penjelasan}
Karena elemen yang paling banyak muncul adalah 2, dengan frekuensi 3, maka derajat array masukan adalah 3.
Subarray terkecil yang memiliki derajat 3 adalah [2, 2, 3, 1, 4, 2] dengan panjang 6.

\end{document}