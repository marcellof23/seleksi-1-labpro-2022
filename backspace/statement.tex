\documentclass{article}

\usepackage{geometry}
\usepackage{amsmath}
\usepackage{graphicx, eso-pic}
\usepackage{listings}
\usepackage{hyperref}
\usepackage{multicol}
\usepackage{fancyhdr}
\pagestyle{fancy}
\fancyhf{}
\hypersetup{ colorlinks=true, linkcolor=black, filecolor=magenta, urlcolor=cyan}
\geometry{ a4paper, total={170mm,257mm}, top=40mm, right=20mm, bottom=20mm, left=20mm}
\setlength{\parindent}{0pt}
\setlength{\parskip}{0.3em}
\renewcommand{\headrulewidth}{0pt}
\rfoot{\thepage}
\lfoot{Seleksi Asisten Laboratorium Pemrograman STEI ITB 2022/2023}
\lstset{
    basicstyle=\ttfamily\small,
    columns=fixed,
    extendedchars=true,
    breaklines=true,
    tabsize=2,
    prebreak=\raisebox{0ex}[0ex][0ex]{\ensuremath{\hookleftarrow}},
    frame=none,
    showtabs=false,
    showspaces=false,
    showstringspaces=false,
    prebreak={},
    keywordstyle=\color[rgb]{0.627,0.126,0.941},
    commentstyle=\color[rgb]{0.133,0.545,0.133},
    stringstyle=\color[rgb]{01,0,0},
    captionpos=t,
    escapeinside={(\%}{\%)}
}

\begin{document}

\begin{center}
    \section*{Backspace} % ganti judul soal

    \begin{tabular}{ | c c | }
        \hline
        Batas Waktu  & 1s \\    % jangan lupa ganti time limit
        Batas Memori & 128MB \\  % jangan lupa ganti memory limit
        \hline
    \end{tabular}
\end{center}

\subsection*{Deskripsi}
Diberikan satu buah string acak S sebagai masukan. String hanya terdiri dari huruf besar alfabet saja. Apabila pada string tersebut ditemukan huruf `B` maka hapus huruf sebelumnya dan hilangkan huruf B tersebut. Buatlah string baru dan cetak hasilnya berdasarkan ketentuan tersebut!

\subsection*{Format Masukan}
Masukan hanya terdiri dari satu baris. Baris pertama adalah sebuah string acak S yang hanya terdiri dari huruf besar alfabet. Masukan boleh kosong.

\subsection*{Format Keluaran}
Keluaran terdiri dari satu baris yang merupakan string S yang sudah dimodifikasi sesuai deskripsi soal. Apabila ternyata huruf 'B' lebih banyak dari huruf lainnya, maka keluaran menjadi kosong.

\subsection*{Batasan Masukan}
\begin{itemize}
 \item $0 \leq |S|  \leq 10^7$
\end{itemize}

\begin{multicols}{2}
\subsection*{Contoh Masukan}
\begin{lstlisting}
SAFDSBFSDBBBHBHHBBJFSKJBKJBKFDSJFBFF
\end{lstlisting}
\null
\columnbreak
\subsection*{Contoh Keluaran}
\begin{lstlisting}
SAFDJFSKKKFDSJFF
\end{lstlisting}
\vfill
\null
\end{multicols}

\subsection*{Penjelasan}
Untuk setiap huruf 'B' yang ditemukan, huruf sebelumnya dihapus dan huruf 'B' tersebut juga dihapus.
\end{document}