\documentclass{article}

\usepackage{geometry}
\usepackage{amsmath}
\usepackage{graphicx, eso-pic}
\usepackage{listings}
\usepackage{hyperref}
\usepackage{multicol}
\usepackage{fancyhdr}
\pagestyle{fancy}
\fancyhf{}
\hypersetup{ colorlinks=true, linkcolor=black, filecolor=magenta, urlcolor=cyan}
\geometry{ a4paper, total={170mm,257mm}, top=40mm, right=20mm, bottom=20mm, left=20mm}
\setlength{\parindent}{0pt}
\setlength{\parskip}{0.3em}
\renewcommand{\headrulewidth}{0pt}
\rfoot{\thepage}
\lfoot{Seleksi Asisten Laboratorium Pemrograman STEI ITB 2022/2023}
\lstset{
    basicstyle=\ttfamily\small,
    columns=fixed,
    extendedchars=true,
    breaklines=true,
    tabsize=2,
    prebreak=\raisebox{0ex}[0ex][0ex]{\ensuremath{\hookleftarrow}},
    frame=none,
    showtabs=false,
    showspaces=false,
    showstringspaces=false,
    prebreak={},
    keywordstyle=\color[rgb]{0.627,0.126,0.941},
    commentstyle=\color[rgb]{0.133,0.545,0.133},
    stringstyle=\color[rgb]{01,0,0},
    captionpos=t,
    escapeinside={(\%}{\%)}
}

\begin{document}

\begin{center}
    \section*{Subsequence Palindrom} % ganti judul soal

    \begin{tabular}{ | c c | }
        \hline
        Batas Waktu  & 2s \\    % jangan lupa ganti time limit
        Batas Memori & 128MB \\  % jangan lupa ganti memory limit
        \hline
    \end{tabular}
\end{center}

\subsection*{Deskripsi}
Diberikan suatu array dengan panjang n, tentukanlah apakah array tersebut memiliki subsequence dengan panjang minimal 3 yang berupa palindrom.

Subsequence dari suatu array a adalah sebuah array dengan elemen-elemen dari a yang mungkin tidak kontigu, tetapi memiliki urutan yang sesuai dengan urutan pada a.

Sebagai contoh, beberapa subsequence dari array a = [1, 2, 3, 4] adalah [1, 2, 3] dan [1, 4]. Sedangkan [4, 1] bukanlah subsequence dari array a.



\subsection*{Format Masukan}
Baris Pertama, berisi bilangan N, yang menyatakan jumlah elemen dalam array.
Baris kedua, berisi N bilangan bulat $E_1, …, E_N$, menyatakan  elemen-elemen dalam array.


\subsection*{Format Keluaran}
Keluaran berupa sebuah karakter Y apabila ditemukan subsequence yang memenuhi, N apabila tidak.

\subsection*{Batasan Masukan}
\begin{itemize}
 \item $1 \leq  N  \leq 1000$
 \item $1 \leq  {E_i}  \leq 50$
\end{itemize}

\linebreak
\begin{multicols}{2}
\subsection*{Contoh Masukan 1}
\begin{lstlisting}
5
1 2 5 3 2


\end{lstlisting}
\null
\columnbreak
\subsection*{Contoh Keluaran 1}
\begin{lstlisting}
Y
\end{lstlisting}
\vfill
\null
\end{multicols}

\subsection*{Penjelasan}
Ditemukan subsequence dengan panjang minimal 3 yang berupa palindrom, yaitu [2, 3, 2].

\end{document}